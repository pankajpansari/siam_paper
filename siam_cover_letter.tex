% SIAM Article Template
\documentclass[review,onefignum,onetabnum]{siamonline171218}

% Information that is shared between the article and the supplement
% (title and author information, macros, packages, etc.) goes into
% ex_shared.tex. If there is no supplement, this file can be included
% directly.

\title{LP-based Submodular Extensions for Marginal Estimation\thanks{Submitted to the editors 5th January, 2019.}}

\author{Pankaj Pansari\thanks{University of Oxford
  (\email{pankaj@robots.ox.ac.uk}, \email{pawan@robots.ox.ac.uk}).}
\and Chris Russell\thanks{University of Surrey
  (\email{crussell@turing.ac.uk}).}
\and M. Pawan Kumar\footnotemark[2]}
 
\begin{document}
\maketitle

Dear editors,

We are submitting this paper as an extension of a paper published in the
proceedings of AISTATS 2018 titled ``Worst-case Optimal Submodular Extensions
for Marginal Estimation". Both papers focus on deriving submodular
extensions which affect the accuracy of a particular variational inference
method for a conditional random field (CRF). The same connection between this
variational method and linear programming (LP) relaxations of the corresponding
MAP inference problem is used in both versions. In the conference paper, we
used this relation to prove the worst-case optimality of a submodular
extension used in literature for Potts model by connecting it to the tightest
Potts relaxation. We also extended the method to hierarchical Potts model by
deriving its worst-case optimal extension. Finally, we provided an efficient
algorithm that made this variational inference method tractable for dense CRFs.

In this journal version, we extend the use of the variational method to
higher-order CRFs. We derive an accurate submodular extension for
a higher-order CRF with diversity-based clique potentials. This potential favours
labelings that use few number of diverse labels in a clique. For this
derivation, we make use of an LP relaxation of this higher-order model. The
tightness property of this relaxation has not been studied in literature, so we
cannot claim worst-case optimality of the submodular extension for this model.

To show the efficacy of our method for the higher-order diversity model, we
carried additional experiments for (a) stereo matching, and (b) semantic
segmentation on the MSRC-21 dataset. For both applications, we used a CRF with
dense pairwise potential plus the diversity-based higher-order potentials.
\end{document}
