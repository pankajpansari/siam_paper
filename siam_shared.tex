% SIAM Shared Information Template
% This is information that is shared between the main document and any
% supplement. If no supplement is required, then this information can
% be included directly in the main document.


\usepackage{amsfonts}
\usepackage{graphicx}
\usepackage{epstopdf}
\usepackage{algorithmic}
\usepackage[utf8]{inputenc} % allow utf-8 input
\usepackage[T1]{fontenc}    % use 8-bit T1 fonts
\usepackage{hyperref}       % hyperlinks
\usepackage{url}            % simple URL typesetting
\usepackage{booktabs}       % professional-quality tables
\usepackage{amsfonts}       % blackboard math symbols
\usepackage{nicefrac}       % compact symbols for 1/2, etc.
\usepackage{microtype}      % microtypography
\usepackage{amsmath}
\usepackage{algorithm}
\usepackage{algorithmic}
%\usepackage{amsthm}
\usepackage{amssymb}
\usepackage{tikz}
%\usetikzlibrary{arrows.meta}
\usetikzlibrary {positioning}
\usepackage{graphicx}
\usepackage{subcaption}
\usepackage{mathtools}
\usepackage[T1]{fontenc}
\usepackage{lmodern}
\usepackage{qtree}
\usepackage[square, numbers]{natbib}
\setlength{\bibsep}{0.0pt}
\usepackage{sidecap}
\usepackage[toc,page]{appendix}
\usepackage{color}
\usepackage{float}
\usepackage{pdfpages}

\mathchardef\mhyphen="2D 

\newcommand{\imagePath}{/home/pankaj/SubmodularInference/pairwise/data/output}
\renewcommand{\qedsymbol}{$\blacksquare$}

\usepackage{titlesec}
\DeclarePairedDelimiter\ceil{\lceil}{\rceil}
\DeclarePairedDelimiter\floor{\lfloor}{\rfloor}
\DeclareMathOperator*{\argmax}{argmax}
\DeclareMathOperator*{\argmin}{argmin}
\newcommand{\mysection}[1]{\vspace{1mm}\section{#1}\vspace{1mm}}
\newcommand{\mysubsection}[1]{\vspace{2mm}\subsection{#1}\vspace{2mm}}
\newcommand{\mysubsubsection}[1]{\vspace{0mm}\subsubsection{#1} ~\\ \vspace{1mm}}
\newcommand{\myparagraph}[1]{\vspace{3mm}\paragraph{#1}}
\newcommand{\mycaption}[1]{\vspace{0mm}\caption{#1}\vspace{0mm}}

% Packages and macros go here
\ifpdf
  \DeclareGraphicsExtensions{.eps,.pdf,.png,.jpg}
\else
  \DeclareGraphicsExtensions{.eps}
\fi

% Prevent itemized lists from running into the left margin inside theorems and proofs
\usepackage{enumitem}
\setlist[enumerate]{leftmargin=.5in}
\setlist[itemize]{leftmargin=.5in}

% Add a serial/Oxford comma by default.
\newcommand{\creflastconjunction}{, and~}

% Used for creating new theorem and remark environments
\newsiamremark{remark}{Remark}
\newsiamremark{mydef}{Definition}
\newsiamremark{prop}{Property}
%\newsiamremark{theorem}{Theorem}
%\newsiamremark{lemma}{Lemma}
%\newsiamremark{proposition}{Proposition}
%\newsiamremark{corollary}{Corollary}
\newsiamremark{hypothesis}{Hypothesis}
\crefname{hypothesis}{Hypothesis}{Hypotheses}
\newsiamthm{claim}{Claim}

% Sets running headers as well as PDF title and authors
\headers{An Example Article}{D. Doe, P. T. Frank, and J. E. Smith}

% Title. If the supplement option is on, then "Supplementary Material"
% is automatically inserted before the title.
\title{LP-based Submodular Extensions for Marginal Estimation\thanks{Submitted to the editors DATE.
\funding{This work was supported by the EPSRC grants EP/P020658/1, TU/B/000048, and EP/P022529/1 and Google Deepmind PhD studentship.}}}

% Authors: full names plus addresses.
\author{Pankaj Pansari\thanks{ University of Oxford (\email{pankaj@robots.ox.ac.uk}).}
\and Chris Russell\thanks{University of Surrey 
  (\email{crussell@turing.ac.uk}).}
 \and M. Pawan Kumar\thanks{University of Oxford 
  (\email{pawan@robots.ox.ac.uk}).}}
  
\usepackage{amsopn}
\DeclareMathOperator{\diag}{diag}


%%% Local Variables: 
%%% mode:latex
%%% TeX-master: "ex_article"
%%% End: 
